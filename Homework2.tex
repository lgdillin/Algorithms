% !TEX TS-program = pdflatex
% !TEX encoding = UTF-8 Unicode

% This is a simple template for a LaTeX document using the "article" class.
% See "book", "report", "letter" for other types of document.

\documentclass[20pt]{article} % use larger type; default would be 10pt

\usepackage[utf8]{inputenc} % set input encoding (not needed with XeLaTeX)

%%% Examples of Article customizations
% These packages are optional, depending whether you want the features they provide.
% See the LaTeX Companion or other references for full information.

%%% PAGE DIMENSIONS
\usepackage{geometry} % to change the page dimensions
\geometry{a4paper} % or letterpaper (US) or a5paper or....
% \geometry{margin=2in} % for example, change the margins to 2 inches all round
% \geometry{landscape} % set up the page for landscape
%   read geometry.pdf for detailed page layout information

\usepackage{graphicx} % support the \includegraphics command and options

% \usepackage[parfill]{parskip} % Activate to begin paragraphs with an empty line rather than an indent

%%% PACKAGES
\usepackage{booktabs} % for much better looking tables
\usepackage{array} % for better arrays (eg matrices) in maths
\usepackage{paralist} % very flexible & customisable lists (eg. enumerate/itemize, etc.)
\usepackage{verbatim} % adds environment for commenting out blocks of text & for better verbatim
\usepackage{subfig} % make it possible to include more than one captioned figure/table in a single float
\usepackage{mathtools}
% These packages are all incorporated in the memoir class to one degree or another...

%%% Other stuff
\DeclarePairedDelimiter\ceil{\lceil}{\rceil}
\DeclarePairedDelimiter\floor{\lfloor}{\rfloor}

%%% HEADERS & FOOTERS
\usepackage{fancyhdr} % This should be set AFTER setting up the page geometry
\pagestyle{fancy} % options: empty , plain , fancy
\renewcommand{\headrulewidth}{0pt} % customise the layout...
\lhead{}\chead{}\rhead{}
\lfoot{}\cfoot{\thepage}\rfoot{}

%%% SECTION TITLE APPEARANCE
\usepackage{sectsty}
\allsectionsfont{\sffamily\mdseries\upshape} % (See the fntguide.pdf for font help)
% (This matches ConTeXt defaults)

%%% ToC (table of contents) APPEARANCE
\usepackage[nottoc,notlof,notlot]{tocbibind} % Put the bibliography in the ToC
\usepackage[titles,subfigure]{tocloft} % Alter the style of the Table of Contents
\renewcommand{\cftsecfont}{\rmfamily\mdseries\upshape}
\renewcommand{\cftsecpagefont}{\rmfamily\mdseries\upshape} % No bold!

%%% END Article customizations

%%% The "real" document content comes below...

\title{Algorithms Homework 2}
\author{Liam Dillingham}
%\date{} % Activate to display a given date or no date (if empty),
         % otherwise the current date is printed 

\begin{document}
\maketitle

\section{Question 6-1}
We can build a heap by repeatedly calling MAX-HEAP-INSERT to insert the elements into the heap. Consider the following variation on the BUILD-MAX-HEAP procedure:

\subsection{Do the procedures BUILD-MAX-HEAP and BUILD-MAX-HEAP' always create the same heap when run on the same input array? Prove that they do, or provide a counterexample.}

I started out just trying a few small examples to see if I could discover a counter example.  And I did. \\
Suppose we have an array $<1,2,3,4,5>$ \\
By using our new algorithm, BUILD-MAX-HEAP', we obtain the results $<5,4,2,1,3>$
Yet using the original algorithm, BUILD-MAX-HEAP, we obtain $<5,4,3,1,2>$ \\
Thus the two do not produce the same heap.

\subsection{Show that in the worst case, BUILD-MAX-HEAP requires $\Theta(n lg(n))$ time to build an n-element heap}

Note that within BUILD-MAX-HEAP calls MAX-HEAPIFY $\floor{\frac{A.length}{2}}$ times. \\ 
$\frac{1}{2}$ is a constant, so we can remove that.  Thus MAX-HEAPIFY is called $\Theta(n)$ times. \\ 

Suppose that for a given subtree $i$, all of its child subtrees are not max heaps.  That is, if we are at the root, then we will have to compare and exchange values for the entire height of the tree.  Recall that the height of a tree is equal to $\floor{lg(n)}$. Thus, worst case, MAX-HEAPIFY will run $\Theta(lg(n))$ times. \\ 

Therefore, since BUILD-MAX-HEAP calls MAX-HEAPIFY $\Theta(n)$ times, and worst-case MAX-HEAPIFY runs at time $\Theta(lg(n))$, then the worst-case performance for BUILD-MAX-HEAP is $\Theta(nlg(n))$  

\newpage
\section{Question 7.2-2}
What is the running time of QUICKSORT when all elements of array A have the same value? \\ \\

We know that the PARTITION function takes $\Theta(n)$ because it completes $n$ comparisons, so we know it takes at least $\Omega(n)$ time. Note that due to the nature of the comparison, $A[j] \leq x$ (From the book's pseudocode), this condition will always be true.  Since this condition is always true, $i$ is always incremented, and thus $A[r-1]$ and $A[r]$ will be swapped.  Since the last and next to last elements are always swapped, each subsequent call to QUICKSORT recieves a subproblem of size $n-1$, and therfore, the runtime is $\Theta(n^{2})$


\section{Question 7.2-3}
Show that the running time of QUICKSORT is $\Theta(n^{2})$ when the array A contains distinct elements and is sorted in decreasing order. \\ \\

Since $A[r]$ is the least element in the array, each call will produce a subproblem of $0$ on the left, and $n-1$ on the right. Since each call to the partition does $n$ comparisons, each partition call costs $\Theta(n)$ time. Also, note that the partition produces a subproblem of size $n-1$.  Thus, QUICKSORT is called $n$ times, and since PARTITION takes $\Theta(n)$ time, then the run time in the worst case is $\Theta(n)$.



\end{document}
