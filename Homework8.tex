% !TEX TS-program = pdflatex
% !TEX encoding = UTF-8 Unicode

% This is a simple template for a LaTeX document using the "article" class.
% See "book", "report", "letter" for other types of document.

\documentclass[20pt]{article} % use larger type; default would be 10pt

\usepackage[utf8]{inputenc} % set input encoding (not needed with XeLaTeX)

%%% Examples of Article customizations
% These packages are optional, depending whether you want the features they provide.
% See the LaTeX Companion or other references for full information.

%%% PAGE DIMENSIONS
\usepackage{geometry} % to change the page dimensions
\geometry{a4paper} % or letterpaper (US) or a5paper or....
% \geometry{margin=2in} % for example, change the margins to 2 inches all round
% \geometry{landscape} % set up the page for landscape
%   read geometry.pdf for detailed page layout information

\usepackage{graphicx} % support the \includegraphics command and options

% \usepackage[parfill]{parskip} % Activate to begin paragraphs with an empty line rather than an indent

%%% PACKAGES
\usepackage{booktabs} % for much better looking tables
\usepackage{array} % for better arrays (eg matrices) in maths
\usepackage{paralist} % very flexible & customisable lists (eg. enumerate/itemize, etc.)
\usepackage{verbatim} % adds environment for commenting out blocks of text & for better verbatim
%\usepackage{subfig} % make it possible to include more than one captioned figure/table in a single float
\usepackage{mathtools}
\usepackage{graphicx} % supports images in latex
% These packages are all incorporated in the memoir class to one degree or another...

\usepackage{graphicx}
\usepackage{subcaption}

%%% Other stuff
\DeclarePairedDelimiter\ceil{\lceil}{\rceil}
\DeclarePairedDelimiter\floor{\lfloor}{\rfloor}

%%% HEADERS & FOOTERS
\usepackage{fancyhdr} % This should be set AFTER setting up the page geometry
\pagestyle{fancy} % options: empty , plain , fancy
\renewcommand{\headrulewidth}{0pt} % customise the layout...
\lhead{}\chead{}\rhead{}
\lfoot{}\cfoot{\thepage}\rfoot{}

%%% SECTION TITLE APPEARANCE
\usepackage{sectsty}
\allsectionsfont{\sffamily\mdseries\upshape} % (See the fntguide.pdf for font help)
% (This matches ConTeXt defaults)

%%% ToC (table of contents) APPEARANCE
\usepackage[nottoc,notlof,notlot]{tocbibind} % Put the bibliography in the ToC
\usepackage[titles,subfigure]{tocloft} % Alter the style of the Table of Contents
\renewcommand{\cftsecfont}{\rmfamily\mdseries\upshape}
\renewcommand{\cftsecpagefont}{\rmfamily\mdseries\upshape} % No bold!

%%% Code syntax highliting
\usepackage{listings}
%\begin{lstlisting}[language=java]
%\end{lstlisting}

%%% graphics path \graphicspath{{./HW5}}

%%% END Article customizations

%%% nice things to keep around

% \noindent\rule{2cm}{0.4pt} 
%%% puts a small horizontal line

% \mathcal{O} 
%%% big O notation

%%% The "real" document content comes below...

\title{Algorithms Homework 8}
\author{Liam Dillingham}
%\date{} % Activate to display a given date or no date (if empty),
         % otherwise the current date is printed 

\begin{document}
\maketitle

\section{Question 22.2-9} 
Let $G = (V, E)$ be a connected, undirected graph. Give an $\mathcal{O}(V + E)$-time algorithm to compute a path in $G$ that traverses each edge in $E$ exactly once in each direction. Describe how you can find your way out of a maze if you are given a large supply of pennies. \\
\noindent\rule{2cm}{0.4pt} \\

After trying many examples on paper, I have come up with the idea that the best way to do this is to perform a depth-first search.  For some edge $e \in E$ connected to our source node $s$, we follow a path from $s$ to its neighbor $v$, each time checking two cases: 

\begin{itemize}
\item Node $v$ has no unvisited edges
\item Node $v$ is the source node $s$
\end{itemize}

Once either of these cases is satisfied, we follow our path in reverse direction until we reach our node $s$ again, thus crossing each edge on our path exactly once in each direction.  Then, we select another unvisited path and do the same until all edges have been visited once in each direction. 

\newpage
\begin{lstlisting}[language=java]
// all colors are initially assumed to be white
// Assume that G.Adj has nodes related to themselves
COMPUTE-PATH(G, s)

   // s has unvisited neighbors
   for each vertex v in G.adj[s] where v.color != black 
      if v == s
	while !STACK.isEmpty()
	   v = STACK.pop()

      // v has no non-visited neighbors
      else if all u in G.adj[v] have u.color == black 
	while !STACK.isEmpty()
	   v = STACK.pop()

      else
         STACK.push(v)
         COMPUTE-PATH(G, v)
// end
\end{lstlisting}

\newpage
\section{Question 22.3-7} 
Rewrite the procedure DFS, using a stack to elimate recursion. \\
\noindent\rule{2cm}{0.4pt} \\

\begin{lstlisting}[language=java]
DFS(G) // no recursion
   for each vertex u in G.V
      u.color = WHITE
      u.pi = NIL
   time = 0
   STACK is an empty stack
   for each vertex u in G.V
      if u.color == WHITE
         STACK.push(u)
         DFS-VISIT(G, u)

DFS-VISIT(G, u)
   while !STACK.isEmpty()
      v = STACK.pop()
      time = time + 1
      v.d = time
      for each w in G.Adj[v]
         if w.color == WHITE
            w.color = GREY
            w.pi = v
            STACK.push(w)
      time = time + 1
      v.f = time
// end
\end{lstlisting}

\newpage
\section{Question 22.3-10} 
Modify the pseudocode for depth-first search so that it prints out every edge in the directed graph $G$, together with its type. Show what modifications, if any, you need to make if $G$ is undirected. \\
\noindent\rule{2cm}{0.4pt} \\

An imporant distinction that will help us determine the type of edge hooks on the color of a node during its transition from discovery to finish.  It is worth noting that a vertex $u$ is WHITE before time $u.d$, GRAY between $u.d$ and $u.f$, and BLACK thereafter.  This means that if we encounter a WHITE node, it has yet to be discovered, and is therefore a tree edge.  In addition, the book states that back edges are GRAY, and either forward or cross-edges are BLACK.  \\

Also from the book, given two BLACK vertices $u$, $v$, and edge $(u,v)$, this edge is a forward edge if $u.d < v.d$ and cross edge if $u.d > v.d$.  With this information, we can construct our algorithm

\begin{lstlisting}[language=java]
DFS(G)
   for each vertex u in G.V
      u.color = WHITE
      u.pi = NIL
   time = 0
   for each vertex u in G.V
      if u.color == WHITE
         DFS-PRINT(G,u)

DFS-PRINT(G, u)
   time = time + 1
   u.d = time
   u.color = GRAY
   for each v in G.Adj[u]
      if v.color == WHITE
         print("Tree Edge")
         v.pi = u
         DFS-PRINT(G, u)
      else if v.color == GRAY
         print("Back Edge")
      
      // Edge must be black
      else if u.d < v.d
         print("Forward edge")
      else
         print("Cross edge")
   u.color = BLACK
   time = time + 1
   u.f = time
// end
\end{lstlisting}

\newpage
If the graph is undirected, then any node can access any node where there is an edge. Fromt Theorem 22.10, In a depth-first search of an undirected graph, every edge is either a tree edge or a back edge.  Thus the second two cases in my pseudocode will simply never be executed, and they can be removed for simplicity, or left for generality.

\section{Question 22.3-12} 
Show that we can use a depth-first search of an undirected graph $G$ to identify the connected components of $G$, and that the depth-first forest contains as many trees as $G$ has connected components.  More precisely, show how to modify depth-first search so that it assigns to each vertex $v$ an integer label $v.cc$ between $1$ and $k$, where $k$ is the number of connected components of $G$, such that $u.cc = v.cc$ if and only if $u$ and $v$ are in the same connected component. \\
\noindent\rule{2cm}{0.4pt} \\

Suppose we have an undirected graph $G$ and a vertex $v \in G$, with $k$ edges connected to other vertices.  Then each vertex connected to $v$ is a root of a subtree in the depth-first forest.  Thus for each root connected to $v$ will have a value $1...k$, and for some root $i$, all vertices connected will also share that value.  The pseudocode for this is shown below:

\begin{lstlisting}[language=java]
CC-DFS(G)
   for each vertex u in G.V
      u.color = WHITE
      u.pi = NIL
   time = 0
   k = 1   // Initial k = 1
   for each vertex u in G.V
      if u.color == WHITE
         u.cc = k
         k = k + 1
         CC-DFS-VISIT(G, u)

CC-DFS-VISIT(G, u)
   time = time + 1
   u.d = time
   u.color = GRAY
   for each v in G.Adj[u]
      v.cc = u.cc
      if v.color == WHITE
         v.pi = u
         CC-DFS-VISIT(G, v)
   u.color = BLACK
   time = time + 1
   u.f = time
//end
\end{lstlisting}

\newpage
\section{Question 24.1-3} 
Given a weighted, directed graph $G = (V, E)$ with no negative-weight cycles, let $m$ be the maximum over all the vertices $v \in V$ of the minimum  number of edges in a shortest path from the source $s$ to $v$.  (Here, the shortest path is by weight, not the number of edges). Suggest a simple change to the Bellman-Ford algorithm that allows it to terminate in $m + 1$ passes, even if $m$ is not known in advance. \\
\noindent\rule{2cm}{0.4pt} \\

Note that due to the convergence property, if the path between two vertices $u$ and $v$ is equal to $\delta(u,v)$, then relaxing all edges between the two vertices will result in their path still being $\delta(u,v)$. That is, it will not change.  We can exploit this property in our code by checking after each iteration to see if the path changes.  This is where the $+ 1$ comes into our run time.  Although it will only take $m$ passes to compute the shortest path, we will need to try again and verify that there is no change.  Thus it will take $m+1$ iterations. \\

We can achieve this by using a sentinel flag, initializing it to TRUE, and using it as a control in a while loop.  We modify our RELAX() method to return TRUE or FALSE if it can find a path shorter than the current one.  The pseudocode containing the modified BELLMAN-FORD() and RELAX() methods is shown below:

\begin{lstlisting}[language=java]
BELLMAN-FORD(G, w, s)
   flag = TRUE // Flag set to check if the shortest path has changed
   INITIALIZE-SINGLE-SOURCE(G, s)

   while flag == TRUE
      flag = FALSE
      for each edge (u,v) in G.E
         flag = RELAX(u, v, w)

RELAX(u, v, w)
   if v.d > u.d + w(u, v)
      v.d = u.d + w(u, v)
      v.pi = u
      return TRUE
   return FALSE
// end
\end{lstlisting}

Observe that we can omit the TRUE/FALSE return values from the BELLMAN-FORD algorithm, as we are given a graph with no negative-weight cycles.

\end{document}




