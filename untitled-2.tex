% !TEX TS-program = pdflatex
% !TEX encoding = UTF-8 Unicode

% This is a simple template for a LaTeX document using the "article" class.
% See "book", "report", "letter" for other types of document.

\documentclass[20pt]{article} % use larger type; default would be 10pt

\usepackage[utf8]{inputenc} % set input encoding (not needed with XeLaTeX)

%%% Examples of Article customizations
% These packages are optional, depending whether you want the features they provide.
% See the LaTeX Companion or other references for full information.

%%% PAGE DIMENSIONS
\usepackage{geometry} % to change the page dimensions
\geometry{a4paper} % or letterpaper (US) or a5paper or....
% \geometry{margin=2in} % for example, change the margins to 2 inches all round
% \geometry{landscape} % set up the page for landscape
%   read geometry.pdf for detailed page layout information

\usepackage{graphicx} % support the \includegraphics command and options

% \usepackage[parfill]{parskip} % Activate to begin paragraphs with an empty line rather than an indent

%%% PACKAGES
\usepackage{booktabs} % for much better looking tables
\usepackage{array} % for better arrays (eg matrices) in maths
\usepackage{paralist} % very flexible & customisable lists (eg. enumerate/itemize, etc.)
\usepackage{verbatim} % adds environment for commenting out blocks of text & for better verbatim
\usepackage{subfig} % make it possible to include more than one captioned figure/table in a single float
% These packages are all incorporated in the memoir class to one degree or another...

%%% HEADERS & FOOTERS
\usepackage{fancyhdr} % This should be set AFTER setting up the page geometry
\pagestyle{fancy} % options: empty , plain , fancy
\renewcommand{\headrulewidth}{0pt} % customise the layout...
\lhead{}\chead{}\rhead{}
\lfoot{}\cfoot{\thepage}\rfoot{}

%%% SECTION TITLE APPEARANCE
\usepackage{sectsty}
\allsectionsfont{\sffamily\mdseries\upshape} % (See the fntguide.pdf for font help)
% (This matches ConTeXt defaults)

%%% ToC (table of contents) APPEARANCE
\usepackage[nottoc,notlof,notlot]{tocbibind} % Put the bibliography in the ToC
\usepackage[titles,subfigure]{tocloft} % Alter the style of the Table of Contents
\renewcommand{\cftsecfont}{\rmfamily\mdseries\upshape}
\renewcommand{\cftsecpagefont}{\rmfamily\mdseries\upshape} % No bold!

%%% END Article customizations

%%% The "real" document content comes below...

\title{Algorithms Homework 1}
\author{Liam Dillingham}
%\date{} % Activate to display a given date or no date (if empty),
         % otherwise the current date is printed 

\begin{document}
\maketitle

\section{Question 4.5-1}

Use the master method to give tight asymptotic bounds for the following recurrences.

\subsection{$T(n) = 2T(n/4) + 1$}

(1) Extract:
\\ \indent $a = 2, b = 4, f(n) = 1$ 
\\ \\
(2) Determine:
\\ \indent $n^{\log_b a} = n^{\log_4 2} = n^{1 / 2}$ 
\\ \\
(3) Compare:
\\ \indent $\lim_{n\to\infty} n^{1 / 2} \geq \lim_{n\to\infty}1$
\\ \\
(4) We have case $f(n) = \mathcal{O}(n^{\log_b a - \epsilon})$
\\ Thus the answer is $T(n) = \Theta(n^{\log_4 2}) =  \Theta(\sqrt{n})$

\subsection{$T(n) = 2T(n/4) +  \sqrt{n}$}

(1) Extract:
\\ \indent $a = 2, b = 4, f(n) = \sqrt{n}$ 
\\ \\
(2) Determine:
\\ \indent $n^{\log_b a} = n^{\log_4 2} = n^{1 / 2}$ 
\\ \\
(3) Compare:
\\ \indent $\lim_{n\to\infty} n^{1 / 2} = \lim_{n\to\infty}\sqrt{n}$
\\ \\
(4) We have case $f(n) = \Theta(n^{\log_b a - \epsilon})$
\\ Thus the answer is $T(n) = \Theta(n^{\log_4 2}\lg n) = \Theta(\sqrt{n}lg(n))$

\newpage
\subsection{$T(n) = 2T(n/4) + n$}

(1) Extract:
\\ \indent $a = 2, b = 4, f(n) = n$ 
\\ \\
(2) Determine:
\\ \indent $n^{\log_b a} = n^{\log_4 2} = n^{1 / 2}$ 
\\ \\
(3) Compare:
\\ \indent $\lim_{n\to\infty} n^{1 / 2} \leq \lim_{n\to\infty}n$
\\ \\
(4) We have case $f(n) = \Omega(n^{\log_b a + \epsilon})$
\\ Thus the answer is $T(n) = \Theta(n)$

\subsection{$T(n) = 2T(n/4) + n^2$}

(1) Extract:
\\ \indent $a = 2, b = 4, f(n) = n^{2}$ 
\\ \\
(2) Determine:
\\ \indent $n^{\log_b a} = n^{\log_4 2} = n^{1 / 2}$ 
\\ \\
(3) Compare:
\\ \indent $\lim_{n\to\infty} n^{1 / 2} \leq \lim_{n\to\infty}n^{2}$
\\ \\
(4) We have case $f(n) = \Omega(n^{\log_b a + \epsilon})$
\\ Thus the answer is $T(n) = \Theta(n^{2})$

\newpage
\section{Question 4-1}
\subsection{$T(n) = 2T(n/2) + n^{4}$}

(1) Extract:
\\ \indent $a = 2, b = 2, f(n) = n^{4}$ 
\\ \\
(2) Determine:
\\ \indent $n^{\log_b a} = n^{\log_2 2} = n^{1}$
\\ \\
(3) Compare:
\\ \indent $\lim_{n\to\infty} n^{1} \leq \lim_{n\to\infty}n^{4}$
\\ \\
(4) We have case $f(n) = \Omega(n^{\log_b a + \epsilon})$
\\ Thus the answer is $T(n) = \Theta(n^{4})$


\subsection{$T(n) = T(7n/10) + n$}

(1) Extract:
\\ \indent $a = 1, b = 10/7, f(n) = n$ 
\\ \\
(2) Determine:
\\ \indent $n^{\log_1.4 1} = n^{0} = 1$
\\ \\ 
(3) Compare:
\\ \indent  $\lim_{n\to\infty}1 \leq \lim_{n\to\infty}n$
\\ \\ 
(4) We have case $f(n) = \Omega(n^{\log_b a + \epsilon})$
\\ Thus the answer is $T(n) = \Theta(n)$

\subsection{$T(n) = 16T(n/4) + n^{2}$}

(1) Extract:
\\ \indent $a = 16, b = 4, f(n) = n^{2}$ 
\\ \\
(2) Determine:
\\ \indent $n^{\log_b a} = n^{\log_4 16} = n^{2}$
\\ \\
(3) Compare:
\\ \indent  $\lim_{n\to\infty} n^{2} = \lim_{n\to\infty}n^{2}$
\\ \\
(4) We have case $f(n) = \Theta(n^{\log_b a - \epsilon})$
\\ Thus the answer is $T(n) = \Theta(n^{\log_4 16}\lg n) = \Theta(n^{2}lg(n))$

\newpage
\subsection{$T(n) = 7T(n/3) + n^{2}$}

(1) Extract:
\\ \indent $a = 7, b = 3, f(n) = n^{2}$ 
\\ \\
(2) Determine:
\\ \indent $n^{\log_b a} = n^{\log_3 7} = n^{1.77}$
\\ \\
(3) Compare:
\\ \indent  $\lim_{n\to\infty} n^{1.77} \leq \lim_{n\to\infty}n^{2}$
\\ \\
4) We have case $f(n) = \Omega(n^{\log_b a + \epsilon})$
\\ Thus the answer is $T(n) = \Theta(n^{2})$

\subsection{$T(n) = 7T(n/2) + n^{2}$}

(1) Extract:
\\ \indent $a = 7, b = 2, f(n) = n^{2}$ 
\\ \\
(2) Determine:
\\ \indent $n^{\log_b a} = n^{\log_2 7} = n^{2.81}$
\\
Note that because we have the form $n^{\log_2 7}$, we can rewrite as $n^{lg 7}$
\\ \\
(3) Compare:
\\ \indent  $\lim_{n\to\infty} n^{lg 7} \geq \lim_{n\to\infty}n^{2}$
\\ \\
4) We have case $f(n) = \Omega(n^{\log_b a + \epsilon})$
\\ Thus the answer is $T(n) = \Theta(n^{lg 7})$

\subsection{$T(n) = 2T(n/4) + \sqrt{n}$}

(1) Extract:
\\ \indent $a = 2, b = 4, f(n) = \sqrt{n}$ 
\\ \\
(2) Determine:
\\ \indent $n^{\log_b a} = n^{\log_4 2} = n^{1/2}$
\\ \\
(3) Compare:
\\ \indent  $\lim_{n\to\infty} n^{1/2} = \lim_{n\to\infty}\sqrt{n}$
\\ \\
4) We have case $f(n) = \Omega(n^{\log_b a - \epsilon})$
\\ Thus the answer is $T(n) = \Theta(n^{\log_4 2}\lg n) = \Theta(\sqrt{n}lg(n))$

\newpage
\subsection{$T(n) = T(n-2) + n^{2}$}

(1) Note that Master's Method cannot be applied here.
\\ \\
$T(n) = T(n-2) + n^{2}$
\\ \\ solve the equation for $n-2$\\ \\
$T(n-2) = T((n-2) - 2) + (n-2)^{2}$
\\ \\ plug back in \\ \\ 
$T(n) = T(n-4) + (n-2)^{2} + n^{2}$
\\ \\ solve for n-4 \\ \\ 
$T(n-4) = T((n-4) - 2) + (n-4)^{2}$ 
\\ \\ and plug back in \\ \\ 
$T(n) = T(n-6) + (n-4)^{2} + (n-2)^{2} + n^{2}$
\\ \\ now solve for n-6 \\ \\
$T(n-6) = T((n-6) - 2) + (n-6)^{2}$
\\ \\ and finally, plug back in \\ \\
$T(n) = T(n-8) + (n-6)^{2} + (n-4)^{2} + (n-2)^{2} + n^{2}$
\\ \\ 
Note that because $n$ is finite, we will eventually have the form: \\ 
$T(n) = 2^{2} + 4^{2} + 6^{2} + ... + (n-4)^{2} + (n-2)^{2} + n^{2}$

$$\sum_{i=0}^{n/2} (n-2i)^2 = 2^{2} + 4^{2} + 6^{2} + ... + (n-4)^{2} + (n-2)^{2} + n^{2}$$

This is very similar to the square pyramidal series \\

$$ n^{2} + (n-1)^{2} + (n-2)^{2} + ... + 1^{2} = \frac{n(n+1)(2n+1)}{6}$$

Since we are incrementing by 2, we are taking half the series.  As $n$ grows without without bound, the two become identical.
Note that by multiplying through by $n$, we achieve\\
$$\frac{2n^{3} + 3n^{2} +n}{6}$$ \\

Since $n^3$ is the leading term, then we have:
$$T(n) = \Theta(n^{3})$$
\end{document}
