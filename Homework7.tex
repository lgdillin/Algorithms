% !TEX TS-program = pdflatex
% !TEX encoding = UTF-8 Unicode

% This is a simple template for a LaTeX document using the "article" class.
% See "book", "report", "letter" for other types of document.

\documentclass[20pt]{article} % use larger type; default would be 10pt

\usepackage[utf8]{inputenc} % set input encoding (not needed with XeLaTeX)

%%% Examples of Article customizations
% These packages are optional, depending whether you want the features they provide.
% See the LaTeX Companion or other references for full information.

%%% PAGE DIMENSIONS
\usepackage{geometry} % to change the page dimensions
\geometry{a4paper} % or letterpaper (US) or a5paper or....
% \geometry{margin=2in} % for example, change the margins to 2 inches all round
% \geometry{landscape} % set up the page for landscape
%   read geometry.pdf for detailed page layout information

\usepackage{graphicx} % support the \includegraphics command and options

% \usepackage[parfill]{parskip} % Activate to begin paragraphs with an empty line rather than an indent

%%% PACKAGES
\usepackage{booktabs} % for much better looking tables
\usepackage{array} % for better arrays (eg matrices) in maths
\usepackage{paralist} % very flexible & customisable lists (eg. enumerate/itemize, etc.)
\usepackage{verbatim} % adds environment for commenting out blocks of text & for better verbatim
%\usepackage{subfig} % make it possible to include more than one captioned figure/table in a single float
\usepackage{mathtools}
\usepackage{graphicx} % supports images in latex
% These packages are all incorporated in the memoir class to one degree or another...

\usepackage{graphicx}
\usepackage{subcaption}

%%% Other stuff
\DeclarePairedDelimiter\ceil{\lceil}{\rceil}
\DeclarePairedDelimiter\floor{\lfloor}{\rfloor}

%%% HEADERS & FOOTERS
\usepackage{fancyhdr} % This should be set AFTER setting up the page geometry
\pagestyle{fancy} % options: empty , plain , fancy
\renewcommand{\headrulewidth}{0pt} % customise the layout...
\lhead{}\chead{}\rhead{}
\lfoot{}\cfoot{\thepage}\rfoot{}

%%% SECTION TITLE APPEARANCE
\usepackage{sectsty}
\allsectionsfont{\sffamily\mdseries\upshape} % (See the fntguide.pdf for font help)
% (This matches ConTeXt defaults)

%%% ToC (table of contents) APPEARANCE
\usepackage[nottoc,notlof,notlot]{tocbibind} % Put the bibliography in the ToC
\usepackage[titles,subfigure]{tocloft} % Alter the style of the Table of Contents
\renewcommand{\cftsecfont}{\rmfamily\mdseries\upshape}
\renewcommand{\cftsecpagefont}{\rmfamily\mdseries\upshape} % No bold!

%%% Code syntax highliting
\usepackage{listings}
%\begin{lstlisting}[language=java]
%\end{lstlisting}

%%% graphics path \graphicspath{{./HW5}}

%%% END Article customizations

%%% nice things to keep around

% \noindent\rule{2cm}{0.4pt} 
%%% puts a small horizontal line

% \mathcal{O} 
%%% big O notation

%%% The "real" document content comes below...

\title{Algorithms Homework 7}
\author{Liam Dillingham}
%\date{} % Activate to display a given date or no date (if empty),
         % otherwise the current date is printed 

\begin{document}
\maketitle

\section{Question 16.1-5} 
Consider a modification to the activity-selection problem in which each activity $a_i$ has, in addition to a start and finish time, a value $v_i$. The objective is no longer to maximize the number of activities scheduled, but instead to maximize the total value of the activities scheduled.  That is, we wish to choose a set $A$ of compatible activities such that $\sum_{a_k \in A}^{} v_k$ is maximized. Give a polynomial-time algorithm for this problem. \\ 
\noindent\rule{2cm}{0.4pt} \\

My algorithm is a modified version of the book's iterative GREEDY-ACTIVITY-SELECTOR($s$, $f$).  Some things I used that may be different is Java's TreeSet and HashMap classes. I built an Activity class to handle all the attributes associated with each activity, and used the TreeSet to contain the output, that is, all the activities that were compatible via their start and end times.\\

Next, I used a HashMap and keyed the Activities on their key.  This makes it easier to fetch activities during the activity selection. \\

Then, for the actual greedy selection algorithm, I iterated through the activity ids, starting with $1$, and going to $n$.  I pushed that id into our result set, and tried to find all compatible activities that came afterward. The result of each iteration of this is a set of compatible activities.  From there, I calculate the total value of these activities, and compare to find the max:

\newpage
\begin{lstlisting}[language=java]
greedyActivityValueSelector(i)
	n = HashMap.size() // Number of elements in the hashmap
	Set s = new Set
	s.add(HashMap(i))

	a = HashMap(i) 
	for j = i to n
		b = HashMap(j) // Get consecutive hash keys
		if b.start >= a.finish
			s.add(b)
			a = b
	return s

max = 0
maxSet = null
for activity i to n:
	Set s = greedyActivityValueSelector(i)

	sum = 0
	for each j in s
		sum = sum + j.value

	if sum > max
		max = sum
		maxSet = s

print(maxSet)

\end{lstlisting}




\end{document}




